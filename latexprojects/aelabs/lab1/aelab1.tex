\documentclass[12pt, a4paper, oneside]{ctexart}
\usepackage{fancyhdr}
\usepackage{amsmath, amsthm, amssymb, bm, graphicx, hyperref, mathrsfs}
\renewcommand\thesubsection{\arabic{subsection}.}
\renewcommand\thesubsubsection{(\arabic{subsubsection})}

\usepackage[left=1in, right=1in, top=1in, bottom=1in]{geometry}

\pagestyle{fancy}
\fancyhf{}
\renewcommand{\headrulewidth}{0pt}
\fancyfoot[C]{\thepage}

\title{\textbf{模拟电子技术——仿真作业1}}
\author{522031910129 张浩宇}
\date{}

\begin{document}
    \maketitle

    \section{仿真电路}

        \begin{figure}[h]
            \centering
            \includegraphics[width=0.8\textwidth]{img/2023-11-06.png}
            \caption{仿真电路}
        \end{figure}

    \section{仿真问题求解}
        \subsection{}
        \subsubsection{直流仿真得静态工作点}
            在LTspice中对电路运行直流仿真,得到静态工作点数据如图2.

            \begin{figure}[h]
                \centering
                \includegraphics[width=0.8\textwidth]{img/2023-11-06 (1).png}
                \caption{直流工作点仿真}
            \end{figure}
            
            可得静态工作点为
            $$
            I_{BQ}\approx5.463\mu\text{A},I_{CQ}\approx1.157 \text{mA},
            $$
            $$
            U_{ECQ}\approx 6.887 \text{V}.
            $$

            则三极管共射电流放大倍数$\beta={ I_{CQ} \over I_{BQ}}\approx212 $.
        \subsubsection{估算静态工作点}
            在直流通路下求解静态工作点.在输入端,
            $$
            3\text{V}=I_{EQ}R_1-U_{EBQ}-I_{BQ}R_3
            $$
            其中
            $$
            I_{EQ}=(1+\beta)I_{BQ},
            $$
            在输出端,
            $$
            15\text{V}-I_{EQ}R_1-U_{ECQ}-I_{CQ}R_2=0,
            $$
            其中
            $$
            I_{CQ}=\beta I_{BQ}.
            $$

            三极管电流放大倍数$\beta$取仿真值$\beta=$,$U_{EBQ}$根据三极管数据手册取$U_{EBQ}=0.65 \text{V}$.
            可得静态工作点为
            $$
            I_{BQ}\approx5.478\mu\text{A},I_{CQ}\approx1.161\text{mA},
            $$
            $$
            U_{ECQ}\approx6.861\text{V}.
            $$
        \subsubsection{比较与分析}  
        比较仿真的计算所得静态工作点,两者较为接近.计算所得$I_{BQ},I_{CQ}$较大,可能的原因是取的$U_{EBQ}$较小.

        由静态工作点数据可得
        $$
        U_{ECQ} > U_{EBQ} >0,
        $$
        表明发射结正偏,集电结反偏,三极管工作在放大状态

        \newpage
        \subsection{}
        \subsubsection{直流扫电压仿真}
        \begin{figure}[h]
            \centering
            \includegraphics[width=0.8\textwidth]{img/2023-11-06 (2).png}
            \caption{直流扫电压仿真}
        \end{figure}

        用LTspice仿真$V4$从0变为15V过程中,$I_c$和$U_c$的变化情况,如图3所示.
        \subsubsection{分析变化情况}
        \begin{figure}[h]
            \centering
            \includegraphics[width=0.8\textwidth]{img/2023-11-06 (7).png}
            \caption{直流扫电压仿真}
        \end{figure}
        用LTspice仿真$V4$从0变为15V过程中,$U_c$,$U_b$和$U_e$的变化情况,如图4所示.由图中可知,5V之前三极管为放大状态,5V之后集电结正偏,三极管进入饱和状态.
        当三极管为放大状态时,在直流通路中有
        $$
        V4=({1+\beta \over \beta}R_1+{1 \over \beta} R_3)I_c+U_{EB},
        $$
        $$
        U_c=I_cR_2,
        $$
        因此$I_c$和$U_c$与V4近似线性关系


        

        

        \subsection{}
        \begin{figure}[h]
            \centering
            \includegraphics[width=0.8\textwidth]{img/2023-11-06 (5).png}
            \caption{时域仿真}
        \end{figure}
        
        \subsubsection{时域仿真}
        
        保持其他参数为初始值,改变$V4$的直流电压为1V,用LTspice进行时域仿真,基极电压和输出电压$VOUT$的变化如图5所示.
        \subsubsection{失真分析}
        由$VOUT$波形可以看出此时出现了失真,失真出现在输入波形正半周.

        此时基极直流偏置电压14V,而发射极偏置电压略小于15V.在基极交流输入波形正半周时,基级电位从14V到15
        V波动,故会出现一段基级电位大于射级电位,此时发射结截止,三极管处于截止状态,发生截止失真.
        \subsection{}
        \subsubsection{直流工作点仿真}
        保持其他参数为初始值,改变$R_2$的电阻为$15\text{k}\Omega$,用LTspice进行直流仿真分析工作状态,如图6所示.
        \begin{figure}[h]
            \centering
            \includegraphics[width=0.8\textwidth]{img/2023-11-06 (6).png}
            \caption{时域仿真}
        \end{figure}

        \subsubsection{工作区域分析}
        由仿真结果,有$U_e>U_b,U_c>U_b$,即三极管发射结集电结均正偏,三极管工作在饱和状态.
        
        \subsection{}
        \subsubsection{直流扫电压仿真}
        \begin{figure}[h]
            \centering
            \includegraphics[width=0.8\textwidth]{img/2023-11-06 (8).png}
            \caption{直流扫电压仿真}
        \end{figure}
        用LTspice仿真$V4$从0变为15V过程中,直流通路$U_c$,$U_b$和$U_e$的变化情况,如图7所示.
        
        图中$U_c$有两个拐点.在第一个拐点之前,$U_e-U_b<U_on$,发射结和集电结均反偏,三极管处于截止状态,视为开路,故集电极电压保持为0.
        第 一个拐点之后,$U_e-U_b>U_on$发射结开启,集电结仍截止,三极管处于放大状态,与2(2)同理,此时$U_c$线性增加.
        第二个拐点处,恰对应$U_c-U_b=U_on$,此后集电结也开启,三极管进入饱和状态,集电极电压与基极电压同步变化,而V4增大,基极电压减小,集电极电压也跟随减小

\end{document}